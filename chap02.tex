%% Copyright (c) 2002, 2010 Sam Williams
%% Copyright (c) 2010 Richard M. Stallman
%% Permission is granted to copy, distribute and/or modify this
%% document under the terms of the GNU Free Documentation License,
%% Version 1.3 or any later version published by the Free Software
%% Foundation; with no Invariant Sections, no Front-Cover Texts, and
%% no Back-Cover Texts. A copy of the license is included in the
%% file called ``gfdl.tex''.

\chapter{2001: Хакерская одиссея}

В двух кварталах к востоку от парка Вашингтон-Сквер стоит здание Уоррена Уивера -- брутальное и внушительное, словно крепость. Здесь располагается факультет информатики Нью-Йоркского Университета. Вентиляционная система, выполненная в промышленном стиле, создаёт вокруг здания сплошную завесу горячего воздуха, равно обескураживая снующих дельцов и слоняющихся бездельников. Если посетителю всё-таки удаётся преодолеть эту линию обороны, его встречает следующий грозный рубеж -- стойка регистрации прямо у единственного входа.

После стойки регистрации градус суровости атмосферы несколько спадает. Но и здесь посетитель то и дело встречает знаки, предупреждающие об опасности незапертых дверей и заблокированных пожарных выходов. Они словно напоминают, что безопасности и осторожности много не бывает даже в ту спокойную эпоху, что закончилась 11 сентября 2001 года.

И знаки эти забавно контрастируют с публикой, заполняющей внутренний зал. Некоторые из этих людей действительно похожи на студентов престижного Нью-Йоркского Университета. Но основная масса больше похожа на взлохмаченных завсегдатаев концертов и клубных выступлений, они словно вышли на свет в перерыве между актами. Эта пёстрая публика так стремительно заполонила здание сегодняшним утром, что местный охранник только махнул рукой и сел смотреть шоу Рики Лейк по телевизору, всякий раз пожимая плечами, когда нежданные посетители обращались к нему с вопросами по поводу некой \enquote{речи}.

Пройдя в аудиторию, посетитель видит того самого человека, который ненароком отправил в аут могучую систему безопасности здания. Это Ричард Мэтью Столлман, основатель проекта GNU, учредитель фонда свободного программного обеспечения, лауреат стипендии Мак-Артура за 1990 год, лауреат премии Грейс Мюррей Хоппер за тот же год, сополучатель премии Такеда в области улучшений экономической и социальной жизни, и просто хакер Лаборатории ИИ. Как гласило объявление, разосланное по множеству хакерских сайтов, включая и официальный \url{http://www.gnu.org} портал проекта GNU, Столлман прибыл на Манхэттен, в свой родной край, чтобы произнести долгожданную речь в противовес кампании, развёрнутой Microsoft против лицензии GNU GPL.

Речь Столлмана посвящалась прошлому и будущему движения свободного ПО. Место было выбрано неслучайно. За месяц до этого старший вице-президент компании Microsoft Крейг Мунди отметился совсем рядом, в Школе бизнеса того же университета. Отметился речью, которая состояла из нападок и обвинений в адрес лицензии GNU GPL. Эту лицензию Ричард Столлман создал после истории с лазерным принтером Xerox 16 лет тому назад в качестве средства борьбы с лицензиями и договорами, которые окутали компьютерную индустрию непроницаемыми завесами секретности и собственничества. Суть GNU GPL в том, что она создаёт общественную форму собственности -- то, что сейчас называется \enquote{цифровым достоянием общества} -- используя юридическую силу авторского права, то есть именно того, против чего направлена. GPL сделала эту форму собственности безвозвратной и неотчуждаемой -- однажды переданный обществу код невозможно отобрать и присвоить. Производные работы, если они используют GPL-код, должны наследовать эту лицензию. Из-за этой особенности критики GNU GPL называют её \enquote{вирусной}, как будто она распространяется на каждую программу, которой только касается. \endnote{Конечно, GPL далеко не такая мощная -- недостаточно просто поместить код в компьютер с GPL-программами, чтобы он стал GPL-кодом}.

\enquote{Сравнение с вирусом это слишком жёстко, -- говорит Столлман, -- куда лучше сравнение с цветами: они распространяются, если вы активно их рассаживаете}.

Если вы хотите узнать больше о лицензии GPL, посетите сайт проекта GNU \url{http://www.gnu.org/copyleft/gpl.html}.

Для высокотехнологичной экономики, которая всё больше зависит от программного обеспечения и всё сильнее привязывается к программным стандартам, GPL стала настоящей \enquote{большой дубинкой}. Даже те компании, что поначалу потешались над ней, называя \enquote{социализмом для программ}, стали признавать преимущества этой лицензии. Ядро Linux, разработанное финским студентом Линусом Торвальдсом в 1991 году, лицензируется под GPL, равно как и большинство компонентов системы: GNU Emacs, GNU Debugger, GNU GCC, и так далее. Все вместе эти компоненты образуют свободную операционную систему GNU/Linux, которая разрабатывается и принадлежит мировому сообществу. Высокотехнологичные гиганты вроде IBM, Hewlett-Packard и Oracle вместо того, чтобы видеть в постоянно растущем свободном ПО угрозу, используют его как основу для своих коммерческих приложений и сервисов. \endnote{Однако то, что эти приложения и сервисы работают на GNU/Linux, ещё не значит, что они являются свободным программным обеспечением. Наоборот, в большинстве своём они имеют собственническую лицензию и уважают вашу свободу не больше, чем Windows. Они могут способствовать успеху GNU/Linux, но точно не способствуют достижению свободы}.

Также свободное ПО стало их стратегическим инструментом в затяжной войне с корпорацией Microsoft, которая доминирует на рынке программ для персональных компьютеров с конца 80-х годов. Обладая самой популярной настольной операционной системой -- Windows -- Microsoft может понести наибольшие потери от распространения GPL в индустрии. Каждая программа в составе Windows защищена авторскими правами и лицензионными соглашениями типа EULA, в результате исполняемые файлы и исходные коды становятся собственническими, лишая пользователей возможности читать и изменять код. Если Microsoft захочет использовать GPL-код в своей системе, ей придётся перелицензировать всю систему под GPL. А это даст конкурентам Microsoft возможность копировать её продукты, улучшать и продавать их, тем самым подрывая саму основу бизнеса компании -- привязку пользователей к её продукции.

Вот откуда растёт обеспокоенность Microsoft широким принятием GPL индустрией. Вот почему недавно Мунди в своей речи обрушился на GPL и открытый код. (Microsoft даже не признаёт термина \enquote{свободное программное обеспечение}, предпочитая использовать в своих нападках выражение \enquote{открытый код}, о котором говорится в \autoref{chapter:open source}. Делается это для того, чтобы сместить внимание общественности от движения за свободное ПО в сторону большей аполитичности). Именно поэтому Ричард Столлман решил сегодня в этом кампусе публично возразить этой речи.

Двадцать лет для индустрии ПО это большой срок. Только подумайте: в 1980 году, когда Ричард Столлман проклинал лазерный принтер Xerox в лаборатории ИИ, Microsoft не была мировым гигантом компьютерной индустрии, она была небольшим частным стартапом. IBM ещё даже не представил свой первый ПК и не взорвал рынок недорогих компьютеров. Не было и многих технологий, которые мы сегодня воспринимаем как должное -- интернета, спутникового телевидения, 32-битных игровых приставок. То же касается и многих компаний, что сейчас \enquote{играют в высшей корпоративной лиге}, вроде Apple, Amazon, Dell -- их либо не было в природе, либо они переживали не лучшие времена. Примеры можно приводить долго.

Среди тех, кто ценит развитие больше свободы, бурный прогресс за столь короткое время приводится в составе аргументов и за, и против GNU GPL. Сторонники GPL обращают внимание на недолгую актуальность компьютерного оборудования. Во избежание риска купить устаревший продукт, потребители стараются выбирать самые перспективные компании. В результате рынок становится ареной, где победитель получает всё. \endnote{Shubha Ghosh, \enquote{Revealing the Microsoft Windows Source Code}, \textit{Gigalaw.com} (January, 2000), \url{http://www.gigalaw.com/}.} Собственническая программная среда, по их словам, приводит к диктатуре монополий и стагнации рынка. Богатые и могущественные компании перекрывают кислород мелким конкурентам и новаторским стартапам.

Их оппоненты утверждают прямо противоположное. По их словам, продажа ПО -- такое же рискованное занятие, как и его производство, если не больше того. Без юридических гарантий, которые обеспечивают собственнические лицензии, у компаний не будет мотивов заниматься разработкой. Особенно актуально это для \enquote{убийственных программ}, создающих совершенно новые рынки. \endnote{\enquote{Убийственные программы} не обязаны быть собственническими. Но вы, наверное, понимаете, что рынок ПО похож на лотерею -- чем больше потенциальная выгода, тем больше людей желает поучаствовать. Хороший разбор \enquote{убийственных программ} можно прочитать в статье: Philip Ben-David, \enquote{Whatever Happened to the \enquote{Killer App}?}, \textit{e-Commerce News} (December 7, 2000), \url{http://www.ecommercetimes.com/story/5893.html}.} И снова на рынке воцаряется застой, инновации идут на убыль. Как сам Мунди заметил в своей речи, \enquote{вирусный} характер GPL \enquote{несёт угрозу} любой компании, которая использует уникальность своего программного продукта в качестве конкурентного преимущества. 

\begin{quote}
Это также подрывает саму основу независимого сектора коммерческого ПО, потому что фактически делает невозможным распространение ПО по модели покупки продукции, а не только оплаты копирования.\endnote{Craig Mundie, \enquote{The Commercial Software Model}, выдержка из стенограммы речи старшего вице-президента Microsoft, произнесённой в Школе бизнеса Нью-Йоркского Университета 3 мая 2001 года, \url{http://www.microsoft.com/presspass/exec/craig/05-03sharedsource.asp}.}
\end{quote}

Успех и GNU/Linux, и Windows последних 10 лет говорит нам, что обе стороны в чём-то правы. Но Столлман и другие адепты свободного ПО считают, что это второстепенный вопрос. Они говорят, что куда важнее не успех свободных или собственнических программ, а их этичность.

Тем не менее, для участников индустрии ПО крайне важно поймать волну. Даже такие могущественные производители, как Microsoft, уделяют много внимания поддержке сторонних разработчиков, чьи приложения, профессиональные пакеты и игры делают платформу Windows привлекательной для потребителей. Ссылаясь на бурное развитие рынка высоких технологий за последние 20 лет, не говоря уже о впечатляющих достижениях его компании за тот же период, Мунди посоветовал слушателям не слишком впечатляться новой модой на свободное ПО:

\begin{quote}
Двадцатилетний опыт показал, что экономическая модель, которая защищает интеллектуальную собственность, и бизнес-модель, которая компенсирует затраты на исследования и разработку, могут создавать впечатляющие экономические блага и широко распространять их.
\end{quote}

На фоне всех этих слов, прозвучавших месяц назад, Столлман готовится к собственной речи, стоя на сцене в аудитории.

Последние 20 лет совершенно изменили мир высоких технологий в лучшую сторону. Ричард Столлман за это время изменился не меньше, но к лучшему ли? Больше нет того худого, чисто выбритого хакера, который когда-то всё своё время проводил перед любимым PDP-10. Теперь вместо него -- грузный мужчина средних лет с длинными волосами и бородой раввина, человек, тратящий всё своё время на переписку по электронной почте, наставления соратников и выступления, подобные сегодняшнему. Одетый в футболку цвета морской волны и штаны из полиэстера, Ричард похож на пустынного отшельника, который только что вышел из пункта Армии Спасения.

В толпе много последователей столлмановских идей и вкусов. Многие пришли с ноутбуками и мобильными модемами, чтобы как можно лучше записать и передать слова Столлмана ждущей интернет-аудитории. Половой состав посетителей очень неравномерен, на каждую женщину приходится 15 мужчин, причём женщины держат в руках мягкие игрушки -- пингвинов, официальных маскотов Linux, и плюшевых медведей.

Волнуясь, Ричард сходит со сцены, садится на стул в первом ряду и принимается набирать команды на ноутбуке. Так проходят 10 минут, и Столлман даже не замечает растущей толпы студентов, профессоров и поклонников, что снуют перед ним между аудиторией и сценой.

Нельзя просто начать говорить, не проделав перед этим декоративных ритуалов академических формальностей, вроде основательного представления докладчика аудитории. Но Столлман выглядит так, что заслуживает не одного, а всех двух представлений. Майк Юретски, содиректор Центра продвинутых технологий Школы бизнеса, взял на себя первое. 

\enquote{Одна из задач университета -- проводить дебаты и всячески способствовать зарождению интересных дискуссий, -- начинает Юретски, -- и наш сегодняшний семинар полностью соответствует этой миссии. По моему мнению, обсуждение открытого кода представляет особенный интерес}.

Прежде чем Юретски успевает сказать ещё хоть слово, Столлман поднимается во весь рост и машет, как стоящий на обочине из-за поломки водитель.

\enquote{Я занимаюсь свободными программами, -- говорит Ричард под растущий смех аудитории, -- открытый код это другое направление}.

Аплодисменты заглушают смех. Аудитория полна столлмановских партизанов, которые знают о его репутации борца за предельно точные формулировки, равно как и об известной ссоре Ричарда со сторонниками открытого кода в 1998 году. Многие из них ждали чего-то подобного, также как поклонники эпатажных звёзд ждут от своих кумиров их коронных выходок.

Юретски спешно заканчивает своё представление и уступает место Эдмонду Шонбергу, профессору факультета информатики Нью-Йоркского Университета. Шонберг -- программист и участник проекта GNU, он прекрасно знаком с картой расположения терминологических мин. Он ловко резюмирует путь Столлмана с точки зрения современного программиста.

\enquote{Ричард -- отличный пример человека, который, работая над малыми проблемами, начал задумываться о проблеме глобальной -- проблеме недоступности исходного кода, -- говорит Шонберг, -- он разработал последовательную философию, под влиянием которой мы пересмотрели наши представления о производстве программного обеспечения, об интеллектуальной собственности, о сообществе разработчиков программ}.\endnote{Будь это сказано сегодня, Столлман возражал бы против термина \enquote{интеллектуальная собственность}, как против плодящего путаницу и несправедливость. Подробности здесь \url{http://www.gnu.org/philosophy/not-ipr.html}.}

Шонберг под аплодисменты приветствует Столлмана. Тот быстро выключает ноутбук, поднимается на сцену и предстаёт перед аудиторией.

Поначалу выступление Ричарда больше походит на стэндап-номер, чем на политическую речь. \enquote{Хочу поблагодарить Microsoft за весомый повод выступить здесь, -- острит он, -- в последние недели я чувствую себя автором книги, которую где-то запретили в рамках произвола}.

Чтобы ввести непосвящённых в курс дела, Столлман проводит краткий ликбез, построенный на аналогиях. Он сравнивает компьютерную программу с кулинарным рецептом. И то, и другое представляет собой полезные пошаговые инструкции о том, как достичь желаемой цели. И то, и другое можно легко изменить в угоду обстоятельствам или своим пожеланиям. \enquote{Вы не обязаны точно следовать рецепту, -- объясняет Столлман, -- вы можете отбросить какие-нибудь ингредиенты или добавить грибов, просто потому, что вы любите грибы. Положить меньше соли, потому что так вам посоветовал доктор -- да всё что угодно}.

Самое важное, по словам Столлмана, то, что программы и рецепты очень легко распространять. Чтобы поделиться с гостем рецептом ужина, достаточно клочка бумаги и пары минут времени. Копирование компьютерных программ требует и того меньше -- всего пары кликов мышью и толики электроэнергии. В обоих случаях дающий человек получает двойную пользу: укрепляет дружбу и повышает шансы, что так же поделятся и с ним.

\enquote{Теперь представьте, что все рецепты представляют из себя чёрный ящик, -- продолжает Ричард, -- вы не знаете, какие там ингредиенты используются, не можете изменить рецепт и поделиться им с другом. Если вы это сделаете, вас назовут пиратом и упрячут в тюрьму на долгие годы. Такой мир вызовет огромное возмущение и неприятие у людей, которые любят готовить и привыкли делиться рецептами. Но именно таков мир собственнических программ. Мир, в котором общественная добропорядочность запрещается и пресекается}.

После этой вводной аналогии Столлман рассказывает историю с лазерным принтером Xerox. Так же, как кулинарная аналогия, история с принтером -- действенный ораторский приём. Похожая на притчу, история о роковом принтере показывает, как быстро всё может измениться в мире программного обеспечения. Возвращая слушателей во времена, что были задолго до покупок в 1 клик на Амазоне, систем Microsoft и баз данных Oracle, Ричард старается донести до аудитории -- каково было иметь дело с программами, которые ещё не были наглухо замурованы под корпоративными логотипами.

Рассказ Столлмана тщательно выверен и отполирован, подобно заключительной речи окружного прокурора в суде. Дойдя до инцидента в Карнеги-Меллон, когда научный сотрудник отказался поделиться исходниками драйвера принтера, Ричард делает паузу.

\enquote{Он предал нас, -- изрекает Столлман, -- но не только нас. Возможно, он предал и тебя тоже}.

На слове \enquote{тебя} Столлман указывает пальцем на ничего не подозревающего слушателя в аудитории. Тот вскидывает брови, вздрагивает от неожиданности, но Ричард уже выискивает взглядом другую жертву среди нервно хихикающей толпы, выискивает медленно и взвешенно. \enquote{И, по-моему, он скорее всего сделал это и с тобой}, -- говорит он, указывая на человека в третьем ряду.

Аудитория уже не хихикает, а смеётся в голос. Конечно, этот жест Ричарда выглядит немного театральным. Тем не менее, историю с лазерным принтером Xerox Столлман заканчивает с пылом настоящего шоумена. \enquote{На самом деле он предал куда больше людей, чем сидит в этой аудитории, не считая тех, кто родился позже 1980 года, -- подытоживает Ричард, вызывая ещё больше смеха, -- просто потому, что он предал всё человечество}.

Дальше он снижает градус драматизма, сообщая:  \enquote{Он сделал это, подписав соглашение о неразглашении}.

Эволюция Ричарда Мэттью Столлмана от разочарованного научного сотрудника к политическому лидеру говорит о многом. О его упрямом характере и впечатляющей воле. О его ясном мировоззрении и отчётливых ценностях, которые помогли ему основать движение за свободное ПО. О его высочайшей квалификации в программировании -- она позволила ему создать ряд важнейших приложений и стать культовой фигурой для многих программистов. Благодаря этой эволюции неуклонно растёт популярность и влияние GPL, и это юридическое новшество многие называют самым главным достижением Столлмана.

Всё это говорит о том, что меняется характер политического влияния -- оно всё сильнее связывается с информационными технологиями и программами, их воплощающими. 

Наверное поэтому звезда Столлмана становится только ярче, в то время как звёзды многих высокотехнологичных гигантов погасли и закатились. С момента запуска проекта GNU в 1984 году, Столлмана и его движение за свободное ПО сначала игнорировали, потом высмеивали, после чего начали унижать и давить валом критики. Но проект GNU смог преодолеть всё это, хоть и не без проблем и периодических стагнаций, и до сих пор предлагает актуальные программы на рынке ПО, который, между прочим, многократно усложнился за эти десятилетия. Успешно развивается и философия, заложенная Столлманом в основу GNU. \endnote{Акроним GNU означает \enquote{GNU's Not Unix} или \enquote{ГНУ -- Не Юникс}}. В другой части своей Нью-Йоркской речи за 29 мая 2001 года, Столлман кратко поведал о происхождении акронима:

\begin{quote}
Мы, хакеры, часто подбираем забавные и даже хулиганские названия для своих программ, потому что называние программ -- одна из составляющих удовольствия от их написания. Также у нас развита традиция использования рекурсивных аббревиатур, которые показывают, что ваша программа в чём-то похожа на уже существующие приложения \ldots Я подыскивал рекурсивную аббревиатуру в форме \enquote{Некая-штука Это Не Юникс}. Я перебрал все буквы алфавита, и ни одна из них не составляла подходящего слова. Я решил сократить фразу до трёх слов, получив таким образом трёхбуквенную аббревиатуру вида \enquote{Некая-штука -- Не Юникс}. Начал перебирать буквы и наткнулся на слово \enquote{GNU}. Вот и вся история.

Хотя Ричард -- поклонник каламбуров, он советует произносить акроним по-английски с отчётливой \enquote{г} в начале, чтобы избежать не только путаницы с названием африканской антилопы гну, но и схожести с английским прилагательным \enquote{new}, т.е. \enquote{новый}. \enquote{Мы работаем над проектом уже пару десятилетий, так что никакой он не новый}, -- шутит Столлман.
\end{quote}

Источник: авторские примечания к стенограмме Нью-Йоркской речи Столлмана \enquote{Свободное ПО: свобода и сотрудничество} за 29 мая 2001 года \url{http://www.gnu.org/events/rms-nyu-2001-transcript.txt}.

Пониманию причин этой востребованности и успешности очень помогает изучение речей и высказываний как самого Ричарда, так и его окружающих, что помогают ему или вставляют палки в колёса. Образ личности Столлмана не нужно переусложнять. Если и есть живой пример старой поговорки \enquote{реальность именно такова, какой выглядит}, то это Столлман.

\enquote{Я думаю, если вы хотите понять Ричарда Столлмана как человека, то вам нужно не анализировать его по частям, а смотреть на него в целом, -- рассуждает Эбен Моглин, юрисконсульт фонда свободного ПО и профессор права Колумбийского Университета, -- все эти эксцентричные моменты, которые многие люди считают чем-то искусственным, наигранным -- на самом деле, искренние проявления личности Ричарда. Он действительно очень сильно разочаровался когда-то, действительно крайне принципиален в этических вопросах и отметает любые компромиссы в главнейших, фундаментальных проблемах. Именно поэтому Ричард сделал всё то, что он сделал}.

Нелегко объяснить, как столкновение с лазерным принтером доросло до схватки с богатейшими корпорациями мира. Для этого нужно вдумчиво изучить причины, по которым вопросы владения программным обеспечением вдруг стали настолько важными. Нужно ближе познакомиться с человеком, который, подобно многим политическим лидерам прошлых времён, понимает, насколько изменчива и податлива память людей. Нужно понимать смысл мифов и идеологических шаблонов, которыми со временем обрастала фигура Столлмана. Наконец, нужно осознавать уровень гениальности Ричарда как программиста, и почему эта гениальность порой терпит фиаско в других областях.

Если попросить самого Столлмана вывести причины его эволюции от хакера до лидера и евангелиста, то он согласится с вышесказанным. \enquote{Упрямство -- моя сильная сторона, -- говорит он, -- большинство людей терпят неудачу в борьбе с большими трудностями просто потому, что сдаются. Я не сдаюсь никогда}.

Также он отдаёт должное слепой случайности. Если бы не история с лазерным принтером Xerox, если бы не ряд личных и идеологических стычек, которые похоронили его карьеру в МТИ, если бы не полдюжины других обстоятельств, пришедшихся ко времени и месту, жизнь Столлмана, по его собственному признанию, была бы совсем другой. Поэтому Столлман благодарит судьбу за то, что она направила его на тот путь, которым он идёт.

\enquote{Просто у меня были нужные способности, -- говорит Ричард в конце своей речи, подводя итог рассказу о запуске проекта GNU, -- никто больше не мог такое сделать, только я. Поэтому я чувствовал, что я избран для этой миссии. Я просто должен был это сделать. Ведь если не я, то кто?\hspace{0.01in}}

\theendnotes
\setcounter{endnote}{0}
